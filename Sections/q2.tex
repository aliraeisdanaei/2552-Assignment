\documentclass[../ass.tex]{subfiles}
\graphicspath{{../Images}}
\addbibresource{../ref.bib}

\begin{document}

\subsection{Part A}
A randomised experiment is one where the researchers treat groups differently based on an artificially designed randomness.
Whereas the natural experiment relies on events occurring in the real, natural world as the random treatment. 
Natural experiments are then obviously less intrusive. 

\subsection{Part B}
The motivation of this experiment is to study the importance of voting for liberals and conservatives. 

A base hypothesis: voter turnout will be lower on days with bad weather, rain, snow, \dots. 
Some ridings in Canada have the same candidates for liberals and conservatives.
The people of closely neighbouring ridings, have similar demographics and vote about the same. 
Demographic information, voter turnouts, vote outcomes, and the weather of a given riding are all public sources of data. 

To measure the importance of voting for different political sides, we can measure the votes for conservatives and liberals on good and bad weather days. 

Of course, there are a lot of confounders. 
However, there can be many points that control for this.
Ridings that have had similar outcomes over the years can be considered to have more weight.
The ratio of voter turnout to the vote outcome can be measured for these ridings. 
Ridings can be compared with during different days of the election. 
Ridings can be measured with neighbouring ridings with similar polical leanings but different weathers.  
The outcome of the general elections can be used to control and cancel out any changes in polical leanings.

\subsection{Part C}
% Counting this is simply counting the number of occurrences of a given event.
% This is useful to know in big data, as the events counted can be categorised differently to support or prove some conclusions..
% The researchers need to count things that are important and significant, not just things that are absent. 
The counting things approach would simply count the number of conservative and liberal voters in comparison with the voter turnout. 
Ridings of lower turnouts may lean towards one election outcome, conversely ridings of higher turnouts may lean towards a different outcome. 

\subsection{Part D}
The second approach is a lot simpler and has fewer confounding factors. 
The natural experiment seeks to measure differences of turnouts due to weather to correlate with certain outcomes.
The second approach simply measures the outcome given the overall turnout. 
I would set up the research with a layered approach; both methods are useful and would give stronger evidences to the conclusions. 


\end{document}