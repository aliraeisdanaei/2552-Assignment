\documentclass[../ass.tex]{subfiles}
\graphicspath{{../Images}}
\addbibresource{../ref.bib}

\begin{document}

\subsection{Part A}
% (a) In 1–2 paragraphs, assess this study using the principles and ethical frameworks in the Ethics chapter % of Bit By Bit.
This study \cite{tagging_banksy} breaches two main principles of ethics described by Salganik \cite{salganik_2019}.
These principles are \textbf{respect for persons} and \textbf{beneficence}.
This study, for the most part, observes justice, the law, and the public interest.

The significance of this research is clear: it provides a method in criminology to identify graffiti related terrorist activities. 
Yet, the application of this methodology in identifying Banksy itself is not very clear. 
What is the benefit of knowing that such a name is behind such artworks?
While there is no clear benefit to identifying Banksy and his approximate location, there are clear harm from this. 
Banksy is an artist that has made an identity through his mystery. 
His livelihood in a way is dependent on his anonymity. 
The harm of the study is to Banksy's artistic career and freedom.
Then there are the harms from people trying to use Banksy's location to injure or harm the person.

There is also the issue of respect for persons, in particular diminishing the autonomy of someone that should be protected. 
It is well known that protecting the address of celebrities, even well liked celebrities is a must, as they are always the target of someone. 
An example of this is John Lennon's murder.

In both ethical frameworks, it is wrong to give out the privacy of a person. 
In a deontological framework, this is a clear breach of principles. 
And in a consequentialist argument, the same methodology of criminology could have been produced by identifying a non-vulnerable suspect or a fake individual. 

\subsection{Part B}
% (b) The authors included the following ethical note at the end of their paper: “The authors are aware of, and respectful of, the privacy of [name redacted] and his relatives and have thus only used data in the public domain. We have deliberately omitted precise addresses.” Does this change your opinion of the paper? If so, how? Do you think the public/private dichotomy makes sense in this case? Discuss in 2–3 sentences.

No, this does not change my opinion. 

Salganik starts off the ethics chapter with \textit{Digital is Different} \cite{salganik_2019}. 
There is significantly more power through the digital world, and much more attention and concern should have been exercised for the privacy of someone like Banksy. 
The methods being published, could easily be enhanced to work out precise locations of Banksy when only public records were used. 

Even if a data is publicly available, it is still not ethical to amplify these results for everyone to take advantage of. 
Furthermore, approximate locations were not omitted in the results. 
The data concludes his whereabouts and his moving from city to city.
This is enough information for people to enhance with other data.

The computational social scientist has a higher ethical obligation than to only stay on the public domain. 
We have seen countless examples of seemingly innocuous studies being exploited in Salganik's book, such as the Netflix Prediction Data, being used to identify the hidden sexuality of given individuals. 

\end{document}