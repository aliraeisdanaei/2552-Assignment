\documentclass[../ass.tex]{subfiles}
\graphicspath{{../Images}}
\addbibresource{../ref.bib}

\begin{document}

The emotional contagion paper ran an experiment of showing more or less positive and negative posts to users \cite{emot_cont}.
One of the limitations of their experiment, among many, was their classification of positive and negative posts. 
They simply measure this by the inclusion f positive or negative words in a given text.
This simplistic approach could be really inaccurate. 

Human computation could have been used to better accurately classify these posts, and improve the ethical implications of the paper. 
Willing participants could have classified given public posts with sensitive information removed. 
A given post should be classified multiple times to increase the accuracy.

With this more accurate measure of positivity of text, we can determine a measure of the ratio of positive to negative posts that a user has seen on their timeline. With this ratio, the positivity of posts from the users can also be observed in a more natural way. 

This method is observational and less intrusive, so it solves a lot of the ethical issues. The only problem is the private posts being read by the participants. However, many of these posts are on public pages, and the users who are friends with the posters can be used as the participants. So the participants who would see these posts could themselves classify the positivity of the posts. Perhaps a slider is placed below random posts to classify its positivity.

% Describe the limitation of the paper that you think crowdsourcing could address, describe your
% crowdsourcing approach, and describe how your approach addresses the limitation.

\end{document}